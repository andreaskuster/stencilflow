\begin{abstract}
Accurate and reliable weather forecast is of vital importance for a broad field of industries and the general public. Highly regular and statically analyzable stencil operators on structured grids are used to numerically solve the partial differential equations of such weather prediction models. This allows optimizations for data re-use while minimizing the high demand of memory bandwidth. Since technology is hitting the power-wall of approximately $1 \frac{Watt}{mm^2}$ for air-cooled CMOS fabrics, while spending a great fraction of the energy for data movement and caching, future high-performance architectures must increase the fraction of energy spent on computations to continue scaling. By implementing custom dataflow architectures on FPGAs, there is a potential to greatly reduce control and data movement overhead. We hope to move a step toward this goal of being more energy efficient compared to the Von-Neuman load/store architecture.\\
In this thesis, we introduce StencilFlow, a framework for mapping stencil programs to FPGAs, offering a complete toolchain from input data analysis, optimization, and simulation, to generation of optimal code for re-programmable devices. We formalize the input program as a directed acyclic graph of streaming modules, and optimize for maximum utilization of on-chip memory and off-chip memory bandwidth. \\
To gain insight into performance metrics and debugging, we develop a model to simulate the stencil program in software. The optimized stencil program is mapped onto FPGA hardware using the DaCe framework. As a realistic use case, we study the COSMO numerical weather forecasting model, currently running on a multi-core and hybrid CPU-GPU architecture. With our framework, high-level users can rapidly implement, optimize, debug, and synthesize arbitrary, large stencil programs onto FPGAs.
\end{abstract}

