\chapter{Introduction}
Accurate and reliable weather forecast is of vital importance for a broad field of industries, as well as the general public. Highly regular and statically analyzable stencil operators \cite{label6} on structured grids  are used to numerically solve the partial differential equations of such weather prediction models. This allows optimizations for data re-use while minimizing the high demand of memory bandwidth. Our collaboration with MeteoSwiss \cite{label37} enables us to apply our theoretical optimization findings to the numerical weather prediction and regional climate model COSMO \cite{label15, label38}. By cooperating closely with Swiss National Supercomputing Center (CSCS) \cite{label39} and the University of Paderborn we gain access to computational hardware resources and intend to figure out together if FPGAs (field-programmable gate arrays) can serve as the next generation accelerator for high-performance weather prediction simulations.


\section{Objectives and Goal}
Nowadays, numerical weather prediction simulations are executed on Von-Neumann architecture \cite{label35} based CPU or GPU clusters \cite{label13}. Since technology is hitting the power-wall of approximately $1 \frac{Watt}{mm^2}$ for air-cooled CMOS fabrics \cite{label1}, high-performance designs should spend a significant amount of that energy for actual computations. Research in the computational field has shown that FPGAs could be a competitive technology in comparison to CPUs \cite{label7}, GPUs \cite{label16} and even ASICs \cite{label8}. By streaming data through the FPGA and therefore greatly reducing control and data movement overhead \cite{label1,label31}, we estimate to be about 8 times (see Appendix A) more energy efficient compared to the load/store architecture. Thus, we seek to implement the numerical model on a state-of-the-art FPGA, the Intel Stratix 10 \cite{label25} using OpenCL \cite{label21} as high-level synthesis (HLS) tool \cite{label18,label36} for increased productivity.


\section{Methods of Investigation / Implementation}
Our theoretical optimization focus lies in the analysis and optimal allocation of resources to reduce the memory bandwidth bottleneck \cite{label2,label3,label28}. By formalizing the input program as a directed graph of streaming modules and formulating it as an optimization problem, we seek to solve it for optimal bandwidth and fast memory usage. To gain insight into performance metrics and for debugging, we develop a model to simulate the stencil program in software. The final step will be to transform this theoretically optimal design onto the FPGA. Studies \cite{label4,label5,label18,label33,label30} have shown that optimization of HLS code is crucial for maximal performance. Therefore, the project is carried out in close collaboration with experts in stencil optimization for different heterogeneous systems (CPU and GPU) \cite{label2,label3,label29, label12,label11,label10,label9} at ETH.


\section{Collaborations}
The outcome of this project highly depends on the symbiosis of partners willing to share their experience, expertise and resources with us. On one hand, we tackle a real-world problem that can help us finding out if our approach can be efficiently scaled up to production scale problems and program sizes. On the other hand, the hardware resources for the synthesis and testing of FPGA designs are very demanding, which is why we have a partner who can provide us with this service. Furthermore, having expert knowledge from other groups involved in high performance FPGA designs helps us a lot to share valuable problem solutions and optimization hints.


\subsection{MeteoSwiss}
MeteoSwiss \cite{label43}, the Federal Office of Meteorology and Climatology located in Zurich, Geneva, Locarno and Payerne is part of the federal administration of Switzerland. Their main task is to constantly create weather forecasts to inform authorities and the general public about upcoming storms and other strong weather phenomena. In addition to that, they provide weather services for civil and the military. \\
MeteoSwiss, together with CSCS are very progressive partners making use of modern, cutting-edge technology. This effort made them the first major national weather service to deploy a GPU-accelerated supercomputer to improve its daily weather forecasts \cite{label44}. \\
We both share the common interest of finding out if FPGAs could be the next technological step forward, which is why we are closely collaborating. This collaboration gives us access to the source code of their current implementation and allows us to directly communicate with experts from MeteoSwiss that are actively involved in the current developments.


\subsection{CSCS}
CSCS \cite{label45}, the Swiss National Supercomputing Center, is the national high-performance computing center of Switzerland. It operates the Piz Daint, one of the world's largest supercomputers (rank 6 in TOP500 at the time of writing, \cite{label42}) in addition to some dedicated computing resources for specific projects and offices of the federal government of Switzerland. \\
Their interest and the associated early purchase of the Stratix 10 FPGA allowed us to do early hands-on test with the hardware and getting valuable insights. In addition to that, high-level synthesis from our high-level OpenCL design to the actual bitstream of the FPGA is a time and resource consuming task. CSCS provides us with Greina and Ault, two large memory cluster computers with integrated FPGA hardware, the required resources for efficient testing and evaluation.


\subsection{University of Paderborn}
The University of Paderborn, Germany, is one of the leading European university in the use of reconfigurable device for high performance applications. With the inauguration of Noctua \cite{label46}, their new cluster computer incorporates 16 nodes with two Intel Stratix 10 FPGA (Bittware 520N) cards each. These cards are interconnected through a optical circuit switch, which gives us the flexibility of potentially splitting the problem and make tests of running it on multiple devices while streaming the necessary data via the dedicated 40Gb/s fiber interconnect. \\
Furthermore, we get expert knowledge from the scientist working in this environment and can share experiences, problems and performance improving design hints. This is especially valuable, since FPGAs are not as main-stream in HPC as CPU and GPUs are. The lack of forum and human resources makes the problem solving more complex. This makes a close collaboration with other people from the same field especially valuable.


\section{Acknowledgement}
I would like to thank my mentor Johannes de Fine Licht for his extraordinary support in this thesis process. Advice given by Professor Torsten Hoefler during the meetings has been of great help in staying on the right track throughout the project. Assistance provided by Carlos Osuna from MeteoSwiss for COSMO related questions was greatly appreciated. I wish to acknowledge the help provided by Hussein Nasser from CSCS by setting up the FPGA cards on Greina and Ault and helping us in case of failure. I am particularly grateful for the assistance given by Tobias Kenter from University of Paderborn in high-level synthesis related questions. Last but not least, Tobias Gysi from ETH Zurich provided me with very valuable inputs from his prior work on the COSMO weather model.
 